\subsection{\textsc{cppslow-fun-10-20}: Monochromatic flat field generation}
\label{cppslow-fun-10-20}

\subsubsection{Requirements}

DMS-REQ-0063,DMS-REQ-0130.

\subsubsection{Test items}

This test will check:

\begin{itemize}

  \item{That a pipeline task (or equivalent tool) exists which generates an
  set of master pure monochromatic flat field images.}

\end{itemize}

\subsubsection{Intercase dependencies}

\hyperref[cppslow-fun-10-05]{CPPSLOW-FUN-10-05},
\hyperref[cppslow-fun-10-15]{CPPSLOW-FUN-10-15},
\hyperref[cppslow-fun-10-25]{CPPSLOW-FUN-10-25}.

\subsubsection{Input specification}

\begin{note}
Detailed specification of the inputs required will require further thought \&
input from the Calibration Scientist; this is a work in progress.
\end{note}

\begin{itemize}

  \item{Monochromatic flat field images;}
  \item{Collimated Beam Projector (CBP) images as specified in \citeds{LDM-151}
  \S4.2.10.}

\end{itemize}

These products should be available within a Butler repository accessible
through the regular LSST data access framework from the system on which the test
is being run.

\subsubsection{Output specification}

\begin{itemize}

  \item{A monochromatic flat field data cube.}

\end{itemize}

These products should be persisted to a Butler repository accessible through
the regular LSST data access framework from the system on which the test is
being run.

\subsubsection{Procedure}

Tasks for assembling, bias correcting and dark correcting the monochromatic
flat field images will be executed from the command line, and the results
persisted to a data repository. These serve as inputs to the monochromatic flat
field data cube production.

The task for generating the monochromatic flat field data cube will be executed
from the command line, and the results persisted to a further data repository.

The Butler will be used to retrieve the flat field data cube from the output
repository, and the contents checked to ensure they are physically plausible
(values TBD by the Calibration Scientist.)
