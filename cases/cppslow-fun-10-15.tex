\subsection{CPPSLOW-FUN-10-15: Illumination correction frame generation}
\label{cppslow-fun-10-15}

\subsubsection{Requirements}

DMS-REQ-0062,DMS-REQ-0130.

\subsubsection{Test items}

This test will check:

\begin{itemize}

  \item{That a pipeline task (or equivalent tool) exists which generates an
  image that corrects for the non-uniform illumination of the flat field
  screen.}

\end{itemize}

\subsubsection{Intercase dependencies}

None.

\subsubsection{Input specification}

\begin{note}
Detailed specification of the inputs required will require further thought \&
input from the Calibration Scientist; this is a work in progress.
\end{note}

\begin{itemize}

  \item{Collimated Beam Projector (CBP) images as specified in \citeds{LDM-151}
  \S4.2.10.}

\end{itemize}

These products should be available within a Butler repository accessible
through the regular LSST data access framework from the system on which the test
is being run.

\subsubsection{Output specification}

\begin{itemize}

  \item{An illumination correction image.}

\end{itemize}

These products should be persisted to a Butler repository accessible through
the regular LSST data access framework from the system on which the test is
being run.

\subsubsection{Procedure}

The task for generating the illumination correction frame will be executed from the
command line, with a configuration appropriate for it to fetch required input
data from the input Butler repository.

The resulting illumination correction image will be persisted to the output
repository. It should be retrieved from the output repository using the Butler
and checked to ensure it contains physically plausible values (TBD by the
Calibration Scientist; ultimately, per \citeds{LSE-61}, it will be verified by
application to operational data).
