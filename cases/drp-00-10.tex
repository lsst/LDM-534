\subsection{DRP-00-10: Data Release Includes Required Data Products}
\label{drp-00-10}

\subsubsection{Requirements}

DMS-REQ-0334, DMS-REQ-0267, DMS-REQ-0268, DMS-REQ-0275, DMS-REQ-0279,
DMS-REQ-0294.

\subsubsection{Test items}
\label{drp-00-10-items}

This test will check that the basic data products which should be in an data
release are generated by execution of the science payload.

These products will include:

\begin{itemize}

  \item{Source catalogs, derived from PVIs and coadded images (DMS-REQ-0267 \& DMS-REQ-0277);}
  \item{Forced source catalogs (DMS-REQ-0268);}
  \item{Object catalogs (DMS-REQ-0275);}
  \item{Processed visit images (PVIs; DMS-REQ-0069);}
  \item{Coadded images (DMS-REQ-0279);}

\end{itemize}

\subsubsection{Intercase dependencies}

\begin{itemize}

  \item{\hyperref[drp-00-00]{DRP-00-00}}

\end{itemize}

\subsubsection{Environmental needs}

\paragraph{Hardware}

The test shall be carried out on a machine with at least 16\,GB of RAM and
multiple CPU cores which has access to the \texttt{/datasets} shared (GPFS)
filesystem at the LSST Data Facility.

\paragraph{Software}

Release 14.0 of the DM Software Stack will be pre-installed (following the
procedure described in \hyperref[drp-00-00]{DRP-00-00}).

\subsubsection{Input specification}

A complete processing of the Hyper Suprime-Cam ``RC2'' dataset, as defined
at ...

\begin{note}
This dataset is currently being defined in DM-11345. We need to make sure the
results of that ticket are clearly published somewhere to complete this test
spec.

\end{note}

... through the DRP Science Payload.

This dataset shall be made available in a standard LSST data repository,
accessible via the ``Data Butler''.

It is not required that all combinations of visit and CCD have been processed
successfully: a number of failures are expected. However, documentation to
describe processing failures should be provided.

\subsubsection{Output specification}

None.

\subsubsection{Procedure}

\begin{itemize}

  \item{The DM Stack shall be initialized using the \texttt{loadLSST} script
  (as described in \hyperref[drp-00-00]{DRP-00-00}).}

  \item{A ``Data Butler'' will be initialized to access the repository.}

  \item{For each of the expected data products types (listed in \S\ref{drp-00-10-items})
  and each of the expected units (PVIs, coadds, etc), the data product will be
  retrieved from the Butler and verified to be non-empty.}

\end{itemize}
