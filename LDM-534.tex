\documentclass[DM,STS,toc]{lsstdoc}
\usepackage{enumitem}
\input meta.tex

\begin{document}

\def\product{LSST Level 2 System}

\setDocCompact{true}

\title[Test Spec for \product]{\product~Test Specification}

\author{Jim Bosch, Hsin-Fang Chiang, Michelle Gower, John D. Swinbank}
\setDocRef{\lsstDocType-\lsstDocNum}
\setDocDate{2018-01-11}

\setDocAbstract {
This document describes the detailed test specification for the \product{}.
}

% Most recent last
\setDocChangeRecord{%
	\addtohist{1.0}{2017-06-30}{Initial release of draft version.}{Swinbank}
	\addtohist{2.0}{2017-07-03}{Updated metadata; no content changes.}{Swinbank}
  \addtohist{3.0}{2018-01-11}{Updated to cover the LDM-503-2 (``HSC Reprocessing'') milestone.}{Bosch, Chiang, Gower, Swinbank}
	\addtohist{3.1}{2018-01-11}{Approved in \jira{RFC-425}}{T.~Jenness}
}

\setDocCurator{John D. Swinbank}
\setDocUpstreamLocation{\url{https://github.com/lsst/ldm-534}}
\setDocUpstreamVersion{\vcsrevision}

\maketitle

\section{Introduction}
\label{sec:intro}

This document specifies the test procedure for the \product{}.

The \product{} is the component of the LSST system which is responsible for
scientific processing leading to:

\begin{itemize}

  \item{Annual data release production;}
  \item{Periodic (re-) generation of calibration products;}
  \item{Periodic (re-) generation of templates for generating difference
  images, to be consumed in the L1 system;}
  \item{Generating QC metrics based on pipeline execution and post-processing of
  scientific data products.}

\end{itemize}

A full description of this product is provided in \S\S7 (which describes the
Data Facility provided execution services) and 13.1 (the science payloads) of
\citeds{LDM-148}.

\subsection{Objectives}
\label{sec:objectives}

This document builds on the description of LSST Data Management's approach to
testing as described in \citeds{LDM-503} to describe the detailed tests that
will be performed on the \product{} as part of the verification of the DM system.

It identifies test designs, test cases and procedures for the tests, and the
pass/fail criteria for each test.

\subsection{Scope}
\label{sec:scope}

This document describes the test procedures for the following components of
the LSST system (as described in \citeds{LDM-148}):

\begin{itemize}

  \item{Services provided by the LSST Data Facility:

    \begin{itemize}
      \item{Template and Calibration Products Production Execution}
      \item{Data Release Production Execution}
      \item{Level 2 Quality Control}
    \end{itemize}
  }

  \item{Science payloads:

    \begin{itemize}
      \item{Annual Calibration Products Production Payload}
      \item{Data Release Production Payload}
      \item{Periodic Calibration Products Production Payload}
      \item{Template Generation Payload}
    \end{itemize}

  }

\end{itemize}

\subsection{Applicable Documents}
\label{sec:docs}

\addtocounter{table}{-1}

\begin{tabular}[htb]{l l}
\citeds{LDM-148} & LSST DM System Architecture \\
\citeds{LDM-151} & LSST DM Science Pipelines Design \\
\citeds{LDM-294} & LSST DM Organization \& Management \\
\citeds{LDM-502} & The Measurement and Verification of DM Key Performance Metrics \\
\citeds{LDM-503} & LSST DM Test Plan \\
\citeds{LSE-61}  & LSST DM Subsystem Requirements \\
\citeds{LSE-163} & LSST Data Products Definition Document \\
\citeds{LSE-180} & Level 2 Photometric Calibration for the LSST Survey \\
\end{tabular}

\subsection{References\label{sect:references}}
\renewcommand{\refname}{}
\bibliography{lsst,refs,books,refs_ads}

%\subsection{Definitions, acronyms, and abbreviations \label{sect:acronyms}} % include acronyms.tex generated by the acronyms.csh (GaiaTools)
%\input{acronyms}


%----------------------------------------------------
% TASK IDENTIFICATION - APPROACH
%----------------------------------------------------
\section{Approach}
\label{sec:approach}

The major activities to be performed are to:

\begin{itemize}

  \item{Compare the design of the Data Release Production payload as
  implemented to the requirements on the outputs of the DM Subsystem as
  defined in \citeds{LSE-63} and \citeds{LSE-163} to demonstrate that all data
  products required by the scientific community will be delivered by the
  system as built.}

  \item{Ensure that all data products included in the DRP payload design are
  correctly produced and persisted appropriately to the LSST Data Backbone
  when executing a data release production.}

  \item{Compare the design of the Calibration Products payloads as implemented
  to the requirements laid down in \citeds{LSE-63}, the overall design
  described in \citeds{LSE-180} and the inputs of the scientific pipeline
  payloads as described in \citeds{LDM-151}.}

  \item{Ensure that all data products included in the CPP payload design are
  correctly produced and persisted appropriately to the LSST Data Backbone
  and/or Calibration Database when executing a calibration products
  production.}

  \item{Compare the implementation of the Template Generation payloads to the
  inputs required by the Alert Production payload as defined in
  \citeds{LDM-151}.}

  \item{Ensure that all data products required by the L1 system are correctly
  produced and persisted appropriately to the LSST Data Backbone when
  executing a template generation production.}

  \item{Demonstrate that QC metrics are properly calculated and transmitted
  during the execution all L2 production types.}

  \item{Demonstrate that post-processing QC analysis of data products can be
  used to identify and report on failures or anomalies in the processing.}

\end{itemize}

\subsection{Tasks and criteria}
\label{sec:tasks}

The following are the major items under test:

\begin{itemize}

  \item{The science payload capable of generating all LSST annual data
  products;}

  \item{Calibration products payloads, run at a variety of cadences, to
  generate calibration products required in the generation of LSST nightly and
  annual data products;}

  \item{The template generation payload capable of generating deep templates
  required for difference imaging in the context of both nightly and annual
  processing.}

  \item{Services capable of scheduling and managing the execution of all of
  the above payloads, marshalling their results, and making them available to
  other parts of the system for analysis or further distribution.}

\end{itemize}

\subsection{Features to be tested}
\label{sec:feat2test}

\begin{itemize}

  \item{Execution of payloads described in \S\ref{sec:tasks};}
  \item{Persistence of all required data products;}
  \item{Scientific fidelity of those data products: do they satisfy the
  requirements described in \citeds{LSE-61}?}

\end{itemize}

\subsection{Features not to be tested}
\label{sec:featnot2test}

This document does not describe facilities for periodically generating or
collecting key performance metrics (KPMs), except insofar as those KPMs are
incidentally measured as part of executing the documented testcases. The KPMs
and the system being used to track KPMs and to ensure compliance with
documented requirements is described in \citeds{LDM-502}.

\subsection{Pass/fail criteria}
\label{sec:passfail}

The results of all tests will be assessed using the criteria described in
\citeds{LDM-503} \S4.

Note that, when executing pipelines, tasks or individual algorithms, any
unexplained or unexpected errors or warnings appearing in the associated log
or on screen output must be described in the documentation for the system
under test. Any warning or error for which this is not the case must be filed
as a software problem report and filed with the DMCCB.

\subsection{Suspension criteria and resumption requirements}
\label{suspension}

Refer to individual test cases where applicable.

\subsection{Naming convention}

All tests are named according to the pattern \textsc{prod-xx-yy} where:

\begin{description}[font=\normalfont\scshape]

  \item{prod}{The product under test. Relevant entries for this document are:
    \begin{description}[font=\normalfont\scshape,topsep=-1.0ex]
      \item[DRP]{The Data Release Production payload and associated service}
      \item[CPP]{The Calibration Products Production payload and associated services}
      \item[TMP]{The Template Generation payload and associated service}
    \end{description}
  }
  \item[xx]{Test specification number (in increments of 10)}
  \item[yy]{Test case number (in increments of 5)}

\end{description}

\section{Test Specification Design}

\subsection{DRP-00: Small Scale Data Release Processing}
\label{drp-00}

\subsubsection{Objective}

This test specification demonstrates the successful execution of a Data
Release Production payload on a relatively small scale based on data from
precursor surveys.

It will demonstrate that:

\begin{itemize}

  \item{Science payload code can be made available on systems managed by the
  LSST Data Facility;}

  \item{The Data Release Production science payload can be executed under the
  control of the Data Release Production Execution service;}

  \item{All required science data products can be collected by the execution
  service and made available for subsequent analysis;}

  \item{The Data Release Production payload generates results broadly
  equivalent to ``native'' reductions of precursor survey data.}

\end{itemize}

Note that this test specification does not extend to demonstrating the
detailed compliance of LSST data products with all \citeds[Science
Requirements Document]{LPM-17} level requirements: such a demonstration would
require carefully curated LSST-like datasets (or simulated data), a detailed
understanding of the LSST system, LSST-like calibration products, etc, which
are assumed not to be available for this test.

\subsubsection{Approach refinements}

The general approach defined in \citeds{LDM-503} is used.

\subsubsection{Test case identification}

\begin{longtable} {|p{0.4\textwidth}|p{0.6\textwidth}|}\hline
\textbf{Test Case}  & \textbf{Description} \\\hline

\hyperref[drp-00-00]{DRP-00-00} & Tests that the Data Release Production science payload can be installed on LSST Data Facility-managed systems.\\\hline
\hyperref[drp-00-05]{DRP-00-05} & Tests the execution of the Data Release Production payload under the control of the Data Release Production Execution Service.\\\hline
\hyperref[drp-00-10]{DRP-00-10} & Tests that required data products are produced by executing the Data Release Production payload. \\\hline

\end{longtable}


% Calibration Products Production test specifications removed to focus this
% document on execution of the LDM-503-2 milestone in late calendar 2017. To
% be reinstated in a future update.
%
%
%\input{specs/cppslow-ver-00.tex}
%\input{specs/cppslow-fun-10.tex}
%\input{specs/cppslow-int-20.tex}
%
%\input{specs/cppyear-int-30.tex}
%
%\input{specs/caldaily-fun-40.tex}
%\input{specs/caldaily-int-50.tex}

\section{Test Case Specification}

\subsection{Preparation}

Before running any test case, the LSST Science Pipelines must be correctly
installed. Follow the procedure described in the
\href{https://pipelines.lsst.io}{Pipelines Documentation}.

% Calibration Products Production test cases removed to focus this document on
% execution of the LDM-503-2 milestone in late calendar 2017. To be reinstated
% in a future update.
%
%\input{cases/cppslow-ver-00-00.tex}
%\input{cases/cppslow-ver-00-05.tex}
%\input{cases/cppslow-ver-00-10.tex}
%
%\input{cases/cppslow-fun-10-00.tex}
%\input{cases/cppslow-fun-10-05.tex}
%\input{cases/cppslow-fun-10-10.tex}
%\input{cases/cppslow-fun-10-15.tex}
%\input{cases/cppslow-fun-10-20.tex}
%\input{cases/cppslow-fun-10-25.tex}
%\input{cases/cppslow-fun-10-30.tex}
%\input{cases/cppslow-fun-10-35.tex}
%
%\input{cases/cppslow-int-20-00.tex}
%\input{cases/cppslow-int-20-05.tex}
%\input{cases/cppslow-int-20-10.tex}
%
%\input{cases/cppyear-int-30-00.tex}
%
%\input{cases/caldaily-fun-40-00.tex}


\appendix

\section{The Hyper Suprime-Cam ``RC'' datasets}

\subsection{RC1}

The original HSC ``release candidate'' dataset was defined as part of testing
release 3.9.0 of the HSC pipeline (derived from the LSST DM Stack). It
consists of 237 visits to the HSC ultra-deep Cosmos field and 83 visits to the
HSC wide survey. Specifically, this includes the following
visits\footnote{Defined using the standard LSST command-line task syntax}:

\subsubsection{Ultra-deep Cosmos}
\label{sec:hscrc1}

\begin{description}

\item[HSC-G]{\hfill \\ 11690..11712:2\^{}29324\^{}29326\^{}29336\^{}29340\^{}29350\^{}29352}
\item[HSC-R]{\hfill \\ 1202..1220:2\^{}23692\^{}23694\^{}23704\^{}23706\^{}23716\^{}23718}
\item[HSC-I]{\hfill \\ 1228..1232:2\^{}1236..1248:2\^{}19658\^{}19660\^{}19662\^{}19680\^{}19682\^{}19684\^{}\\19694\^{}19696\^{}19698\^{}19708\^{}19710\^{}19712\^{}30482..30504:2}
\item[HSC-Y]{\hfill \\ 274..302:2\^{}306..334:2\^{}342..370:2\^{}1858..1862:2\^{}1868..1882:2\^{}11718..11742:2\^{}22602..\\22608:2\^{}22626..22632:2\^{}22642..22648:2\^{}22658..22664:2}
\item[HSC-Z]{\hfill \\ 1166..1194:2\^{}17900..17908:2\^{}17926..17934:2\^{}17944..17952:2\^{}17962\^{}28354..28402:2}
\item[NB0921]{\hfill \\ 23038..23056:2\^{}23594..23606:2\^{}24298..24310:2\^{}25810..25816:2}

\end{description}

\subsubsection{Wide}

\begin{description}

\item[HSC-G]{\hfill \\ 9852\^{}9856\^{}9860\^{}9864\^{}9868\^{}9870\^{}9888\^{}9890\^{}9898\^{}9900\^{}9904\^{}9906\^{}9912\^{}11568\^{}\\11572\^{}11576\^{}11582\^{}11588\^{}11590\^{}11596\^{}11598}
\item[HSC-R]{\hfill \\ 11442\^{}11446\^{}11450\^{}11470\^{}11476\^{}11478\^{}11506\^{}11508\^{}11532\^{}11534}
\item[HSC-I]{\hfill \\ 7300\^{}7304\^{}7308\^{}7318\^{}7322\^{}7338\^{}7340\^{}7344\^{}7348\^{}7358\^{}7360\^{}7374\^{}7384\^{}7386\^{}\\19468\^{}19470\^{}19482\^{}19484\^{}19486}
\item[HSC-Y]{\hfill \\ 6478\^{}6482\^{}6486\^{}6496\^{}6498\^{}6522\^{}6524\^{}6528\^{}6532\^{}6544\^{}6546\^{}6568\^{}13152\^{}13154}
\item[HSC-Z]{\hfill \\ 9708\^{}9712\^{}9716\^{}9724\^{}9726\^{}9730\^{}9732\^{}9736\^{}9740\^{}9750\^{}9752\^{}9764\^{}9772\^{}9774\^{}\\17738\^{}17740\^{}17750\^{}17752\^{}17754}

\end{description}

\end{document}
